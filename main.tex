\PassOptionsToPackage{dvipsnames}{xcolor}

\documentclass[10pt,a4paper,ragged2e,withhyper]{altacv}

\newenvironment{sloppypar*}{\sloppy\ignorespaces}{\par}

\geometry{left=1.2cm,right=1.2cm,top=1cm,bottom=1cm,columnsep=0.75cm}

\usepackage{paracol}

\ifxetexorluatex
  \setmainfont{Roboto Slab}
  \setsansfont{Lato}
  \renewcommand{\familydefault}{\sfdefault}
\else
  \usepackage[rm]{roboto}
  \usepackage[defaultsans]{lato}
  \renewcommand{\familydefault}{\sfdefault}
\fi

\ifdarkmode%
  \definecolor{PrimaryColor}{HTML}{C69749}
  \definecolor{SecondaryColor}{HTML}{D49B54}
  \definecolor{ThirdColor}{HTML}{1877E8}
  \definecolor{BodyColor}{HTML}{ABABAB}
  \definecolor{EmphasisColor}{HTML}{ABABAB}
  \definecolor{BackgroundColor}{HTML}{191919}
\else%
  \definecolor{PrimaryColor}{HTML}{001F5A}
  \definecolor{SecondaryColor}{HTML}{0039AC}
  \definecolor{ThirdColor}{HTML}{F3890B}
  \definecolor{BodyColor}{HTML}{666666}
  \definecolor{EmphasisColor}{HTML}{2E2E2E}
  \definecolor{BackgroundColor}{HTML}{E2E2E2}
\fi%

\colorlet{name}{PrimaryColor}
\colorlet{tagline}{SecondaryColor}
\colorlet{heading}{PrimaryColor}
\colorlet{headingrule}{ThirdColor}
\colorlet{subheading}{SecondaryColor}
\colorlet{accent}{SecondaryColor}
\colorlet{emphasis}{EmphasisColor}
\colorlet{body}{BodyColor}
\pagecolor{BackgroundColor}

\renewcommand{\namefont}{\Huge\rmfamily\bfseries}
\renewcommand{\personalinfofont}{\small\bfseries}
\renewcommand{\cvsectionfont}{\LARGE\rmfamily\bfseries}
\renewcommand{\cvsubsectionfont}{\large\bfseries}

\renewcommand{\itemmarker}{{\small\textbullet}}
\renewcommand{\ratingmarker}{\faCircle}


\begin{document}
\name{Andres Kenichi Koga Nakay}
% \tagline{Test Developer}
% \photoL{4cm}{john-doe}

\personalinfo{
	\email{akoga@pucp.edu.pe}
	\phone{+51 964322765}
	\location{Lima, Perú}\\
	%\homepage{nicolasomar.me}
	%\medium{nicolasomar}
	\NewInfoField{github}{\faGithub}[https://github.com/]
	\github{KKogaa}
	\printinfo{\faLinkedin}{@Koga}[https://www.linkedin.com/in/andres-kenichi-koga-nakay-14b11b219/]
}

\makecvheader

\columnratio{0.25}

\begin{paracol}{2}
	% ----- TECH STACK -----
	\cvsection{Backend}
	\begin{sloppypar*}
		\medskip
		\cvtags{Spring Boot, Flask, FastAPI, Gin, React, NestJs, Postgresql, MongoDB, MySQL,
			JUnit 5, Jest, Pytest, Pandas, Pytorch, Docker, AWS, Clean Code, Clean Architecture}
		\medskip
	\end{sloppypar*}
	% ----- TECH STACK -----

	\cvsection{Frontend}
	\begin{sloppypar*}
		\medskip
		\cvtags{React, Tailwind}
		\medskip
	\end{sloppypar*}
	% ----- TECH STACK -----

	% ----- LEARNING -----
	\cvsection{Lenguajes de Programación}
	\begin{sloppypar}
		\cvtags{Python, Java, Javascript, Typescript, Go}
		\medskip
	\end{sloppypar}
	% ----- LEARNING -----

	% ----- CERTIFICATES -----
	\cvsection{Certificados}
	Cambridge English FCE

	DELF A2

	AWS Academy Cloud Foundations

	% \divider

	% ----- MOST PROUD -----
	\cvsection{Logros}
	2do puesto en Hackathon de ScientOne
	\divider
	% ----- MOST PROUD -----
	\newpage

	\switchcolumn

	% ----- ABOUT ME -----
	\cvsection{Acerca de mí}
	\begin{quote}
		Graduado en Ingeniería Informática, con experiencia y una fuerte motivación en el desarrollo web, arquitec-
		tura de sistemas y aprendizaje máquina. Mi pasión radica en contribuir al desarrollo de soluciones innovadoras
		que impulsen la eficiencia, mejoren las experiencias de usuario y fomenten el crecimiento empresarial.
	\end{quote}
	% ----- ABOUT ME -----

	% ----- EXPERIENCE -----
	\cvsection{Experiencia laboral}
	\cvevent{Backend Developer}{TiendaDa}{04/2022 -- 12/2022}{Lima, Perú}
	\begin{itemize}
		\item Análisis, diseño y desarrollo de soluciones para un e-commerce.
		\item Uso de Flask y Spring Boot para el desarrollo de microservicios.
		\item Implementación con programación reactiva para mejorar la escalabilidad y rendimiento de la aplicación.
	\end{itemize}
	\divider

	\cvevent{Fullstack Developer}{PUCP}{04/2022 -- 12/2022}{Lima, Perú}
	\begin{itemize}
		\item Análisis, diseño y desarrollo para una plataforma de apoyo a los estudiantes.
		\item Implementación de funcionalidades backend utilizando Spring Boot.
		\item Implementación de funcionalidades frontend utilizando React.
		\item Despligue de los microservicios en AWS.
	\end{itemize}
	% ----- EXPERIENCE -----

	% ----- EDUCATION -----
	\cvsection{Educación}
	\cvevent{Ingeniería Informática}{Pontificia Universidad Católica del Perú }{2016 -- 2022}{Lima, Perú}
	\divider
	% ----- EDUCATION -----

	% ----- PROJECTS -----
	\cvsection{Proyectos}
	\cvevent{Motor de búsqueda de imágenes}{\cvreference{\faGithub}{https://github.com/user/repo}\cvreference{}{}}{}{}
	{
		Desarrollo de un motor de búsqueda que permite realizar búsquedas de similitud sobre una colección de imágenes. Permite realizar consultas utilizando textos o imágenes.
	}
	\divider

	\cvevent{Modelo de comprensión lectora}{\cvreference{\faGithub}{https://github.com/user/repo}\cvreference{}{}}{}{}
	{
		Desarrollo de un modelo BERT fine-tuned para resolver problemas de comprensión lectora en el lenguaje español.
	}
	\divider

	\cvevent{Chatbot}{\cvreference{\faGithub}{https://github.com/user/repo}\cvreference{}{}}{}{}
	{
		Implementación de un chatbot que utiliza el modelo BERT para reconocer las intenciones de los usuarios y por ello puede responder a consultas automáticamente.
	}
	\divider

	% \cvevent{Project 3 }{\cvreference{\faGitlab}{https://gitlab.com/user/repo}\cvreference{| \faGlobe}{https://project-demo.com/}}{Mm YYYY -- Mm YYYY}{}
	% \begin{itemize}
	% 	\item Item 1
	% 	\item Item 2
	% \end{itemize}

	% ----- PROJECTS -----
\end{paracol}
\end{document}
